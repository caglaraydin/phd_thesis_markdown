
% Dizin Ayarları

% \def\contentsname{\empty}
% \setcounter{tocdepth}{6}
\usepackage[dotinlabels]{titletoc}
% \contentsmargin{2em}
\titlecontents{chapter}[0em]{\singlespace}{\makebox[4em][l]{\thecontentslabel.\quad\enspace}}{}{\titlerule*[0.75pc]{.}\contentspage}

% Sections
\dottedcontents{section}[4pc]{\singlespace}{4pc}{0.75pc}% %
\dottedcontents{subsection}[4pc]{\singlespace}{4pc}{0.75pc}% %
\dottedcontents{subsubsection}[4em]{\singlespace}{4em}{0.75pc}% %

% Dizinlere Sayfa No yazısı eklemek için
\addtocontents{toc}{~\hfill\textbf{\underline{Sayfa No}}\par}
\addtocontents{lof}{~\hfill\textbf{\underline{Sayfa No}}\par}
\addtocontents{lot}{~\hfill\textbf{\underline{Sayfa No}}\par}

% Dizin başlıklarını düzeltmek için
\setlength{\cftbeforetoctitleskip}{-8mm}
\renewcommand{\contentsname}{\hfill\normalfont\normalsize\textbf{İÇİNDEKİLER}\hfill}
\renewcommand{\cftaftertoctitle}{\hfill}
\setlength{\cftaftertoctitleskip}{28pt}

\setlength{\cftbeforeloftitleskip}{-8mm}
\renewcommand{\listfigurename}{\hfill\normalfont\normalsize\textbf{ŞEKİLLER DİZİNİ}\hfill}
\setlength{\cftafterloftitleskip}{28pt}

\setlength{\cftbeforelottitleskip}{-8mm}
\renewcommand{\listtablename}{\hfill\normalfont\normalsize\textbf{TABLOLAR DİZİNİ}\hfill}
\setlength{\cftafterlottitleskip}{28pt}

% SAYFA NUMARALARI
\usepackage{fancyhdr}
\fancypagestyle{main}{              % Başlıklar
    \fancyhf{}
    \fancyhead[C]{\thepage}
    \setlength{\headheight}{15pt}
    \renewcommand{\headrulewidth}{0.0pt}
    \pagenumbering{arabic}
    \setcounter{page}{1}
}
\pagestyle{plain}{                  % Ön Sayfalar
    \pagenumbering{Roman}
}


% Following package is used to add background image to front page
\usepackage{wallpaper}

% Table package
\usepackage{ctable}% http://ctan.org/pkg/ctable

% Deal with 'LaTeX Error: Too many unprocessed floats.'
\usepackage{morefloats}
% or use \extrafloats{100}
% add some \clearpage

% % Chapter header
% \usepackage{titlesec, blindtext, color}
% \definecolor{gray75}{gray}{0.75}
% \newcommand{\hsp}{\hspace{20pt}}
% \titleformat{\chapter}[hang]{\Huge\bfseries}{\thechapter\hsp\textcolor{gray75}{|}\hsp}{0pt}{\Huge\bfseries}

% Fonts and typesetting
\setmainfont[Scale=1.0]{Times New Roman}
\setsansfont[Scale=1.0]{Verdana}

% FONTS
\usepackage{xunicode}
\usepackage{xltxtra}
\defaultfontfeatures{Mapping=tex-text} % converts LaTeX specials (``quotes'' --- dashes etc.) to unicode
% \setromanfont[Scale=1.01,Ligatures={Common},Numbers={OldStyle}]{Palatino}
% \setromanfont[Scale=1.01,Ligatures={Common},Numbers={OldStyle}]{Adobe Caslon Pro}
%Following line controls size of code chunks
% \setmonofont[Scale=0.9]{Monaco}
%Following line controls size of figure legends
% \setsansfont[Scale=1.2]{Optima Regular}

% CODE BLOCKS
% \usepackage[utf8]{inputenc}
\usepackage{listings}
\usepackage{color}

% JAVA CODE BLOCKS
\definecolor{backcolour}{RGB}{242,242,242}
\definecolor{javared}{rgb}{0.6,0,0}
\definecolor{javagreen}{rgb}{0.25,0.5,0.35}
\definecolor{javapurple}{rgb}{0.5,0,0.35}
\definecolor{javadocblue}{rgb}{0.25,0.35,0.75}

\lstdefinestyle{javaCodeStyle}{
  language=Java,                         % the language of the code
  backgroundcolor=\color{backcolour},    % choose the background color; you must add \usepackage{color} or \usepackage{xcolor}
  basicstyle=\fontsize{10}{8}\sffamily,
  breakatwhitespace=false,
  breaklines=true,
  keywordstyle=\color{javapurple}\bfseries,
  stringstyle=\color{javared},
  commentstyle=\color{javagreen},
  morecomment=[s][\color{javadocblue}]{/**}{*/},
  captionpos=t,                          % sets the caption-position to bottom
  frame=single,                          % adds a frame around the code
  numbers=left,
  numbersep=10pt,                         % margin between number and code block
  keepspaces=true,                       % keeps spaces in text, useful for keeping indentation of code (possibly needs columns=flexible)
  columns=fullflexible,
  showspaces=false,                      % show spaces everywhere adding particular underscores; it overrides 'showstringspaces'
  showstringspaces=false,                % underline spaces within strings only
  showtabs=false,                        % show tabs within strings adding particular underscores
  tabsize=2                              % sets default tabsize to 2 spaces
}

%Attempt to set math size
%First size must match the text size in the document or command will not work
%\DeclareMathSizes{display size}{text size}{script size}{scriptscript size}.
\DeclareMathSizes{12}{13}{7}{7}

% ---- CUSTOM AMPERSAND
% \newcommand{\amper}{{\fontspec[Scale=.95]{Adobe Caslon Pro}\selectfont\itshape\&}}

% Başlıklar
\usepackage{titlesec}
\titleformat{name=\chapter,numberless}[block]           % Ön sayfaların başlık ayarı
{\normalfont\normalsize\bfseries\centering}{}{0pt}{}
\titlespacing{name=\chapter,numberless}{0mm}{-8mm}{28pt}

\titleformat{\chapter}[hang]                     % \titleformat{⟨command⟩}[⟨shape⟩]{⟨format⟩}{⟨label⟩}{⟨sep⟩}{⟨before-code⟩}[⟨after-code⟩]
{\normalfont\normalsize\bfseries}{\thechapter.}{1ex}{}
\titlespacing{\chapter}{10mm}{2mm}{28pt}          %\titlespacing*{⟨command⟩}{⟨left⟩}{⟨before-sep⟩}{⟨after-sep⟩}[⟨right-sep⟩]
\titlelabel{\thetitle.\quad}                        % Section başlıklarında rakamdan sonra nokta koymak için
% \assignpagestyle{\chapter}{empty}                   % Chapter larda sayfa numarasını iptal etmek için

\titleformat{\section}[hang]
{\normalfont\normalsize\bfseries\setstretch{0.1}}{\thesection.}{1ex}{}
\titlespacing{\section}{10mm}{40pt}{21pt}

\titleformat{\subsection}[hang]
{\normalfont\normalsize\bfseries\setstretch{0.1}}{\thesubsection.}{1ex}{}
\titlespacing{\subsection}{10mm}{40pt}{21pt}

\titleformat{\subsubsection}[hang]
{\normalfont\normalsize\bfseries\setstretch{0.1}}{\thesubsubsection.}{1ex}{}
\titlespacing{\subsubsection}{10mm}{40pt}{21pt}

% \titleformat{\paragraph}[hang]
% {\normalfont\normalsize\bfseries\setstretch{0.1}}{\thesubsection.}{1ex}{}
% \titlespacing{\paragraph}{10mm}{40pt}{21pt}


% Set figure legends and captions to be smaller sized sans serif font
\usepackage[singlelinecheck=false, format=hang, justification=justified, font=normalsize, labelsep=period]{caption}
\usepackage{siunitx}
\captionsetup[table]{name=Tablo}
\captionsetup[figure]{name=Şekil}
% \newlength{\mylen}

\addtolength{\cftfignumwidth}{30pt}% More space
\addtolength{\cfttabnumwidth}{30pt}% More space
\DeclareCaptionListFormat{figprefix}{#1\figurename~#2}
\DeclareCaptionListFormat{tabprefix}{#1\tablename~#2}
\captionsetup[figure]{listformat=figprefix}
\captionsetup[table]{listformat=tabprefix}



% \renewcommand{\cftfigpresnum}{\figurename\enspace}
% \renewcommand{\cftfigaftersnum}{:}
% \settowidth{\mylen}{\cftfigpresnum\cftfigaftersnum}
% \addtolength{\cftfignumwidth}{\mylen}

% \renewcommand{\cfttabpresnum}{\tablename\enspace}
% \renewcommand{\cfttabaftersnum}{:}
% \settowidth{\mylen}{\cfttabpresnum\cfttabaftersnum}
% \addtolength{\cfttabnumwidth}{\mylen}


% \usepackage{etoolbox}
% \makeatletter
% \patchcmd{\@caption}{\csname the#1\endcsname}{\csname fnum@#1\endcsname}{}{}
% \renewcommand*\l@figure{\@dottedtocline{1}{1.5em}{4.5em}} % default for 3rd arg: 2.3em
% \let\l@table\l@figure % as in article.cls
% \makeatother


% Satır aralığı ayarı
\usepackage{setspace}
% \onehalfspacing
% \doublespacing
% \raggedbottom
\setstretch{1.44}


% Sayfa yapısı
\usepackage[top=3cm,bottom=2.5cm,left=3cm,right=2.5cm]{geometry}
% \usepackage{showframe}


% Parağraf ayarları
\usepackage{indentfirst}        % Başlık sonrası ilk parağrafın çekmesi
\setlength{\parindent}{10mm}    % Parağraf başlarının çekme mesafesi
\setlength{\parskip}{0pt}       % Parağraflar arası mesafe http://texblog.org/2012/11/07/correctly-typesetting-paragraphs-in-latex/
\hyphenpenalty=100000           % Kelimelerde hecelemeyi engellemek için değeri büyüt

% Set colour of links to black so that they don't show up when printed
% \usepackage{hyperref}
\hypersetup{colorlinks=false, linkcolor=black}

% Tables
\usepackage{booktabs}
\usepackage{threeparttable}
\usepackage{array}
\newcolumntype{x}[1]{%
>{\centering\arraybackslash}m{#1}}%

% Allow for long captions and float captions on opposite page of figures
% \usepackage[rightFloats, CaptionBefore]{fltpage}

% Don't let floats cross subsections
% \usepackage[section,subsection]{extraplaceins}

% Yatay sayfa kullanabilmek için lscape paketi
\usepackage{lscape}
% \begin{landscape} \end{landscape} arasına koyulanı yatay olarak sayfaya yerleştiriyor.